\section{Felder}

\subsection{Elektrostatik}
ist ein wirbelfreies Feld. Elek. Ladungen sind Quellen des Feldes.
\begin{align*}
    \opdiv \vec{D} = \nabla \cdot \vec{D}  &= \rho       \qquad          \vec{D} = \varepsilon \vec{E}\\
    \oprot \vec{E} = \nabla \times \vec{E} &= 0 \quad = \oprot \opgrad E\\
    \vec{E} &=- \opgrad \varphi
\end{align*}
\subsubsection{Potetialgleichung}
\[
    \opdiv \opgrad = - \dfrac{\rho}{\varepsilon}
\]
\paragraph{$\qquad \Rightarrow$ Poisson-Gleichung}
mit $\rho = 0 \Rightarrow$ \textbf{Laplace-Gleichung}
\begin{align*}
    \Delta \varphi + \underbrace{ \dfrac{\opgrad \varepsilon \cdot \opgrad \varphi}{\varepsilon}}_{= 0\textit{, wenn homogen}}       
                                &= - \dfrac{\rho (x, y, z)}{\varepsilon}\\
    \frac{d^2 \varphi}{d x^2} + \frac{d^2 \varphi}{d y^2} + \frac{d^2 \varphi}{d z^2}
                                &= - \dfrac{\rho (x, y, z)}{\varepsilon}
\end{align*}