\section{Maxwell’schen Gleichungen}

\subsection{Intergralform \rn{1}, \rn{2}}
Gauß'sches Gesetz\\
Induktionsgesetz\\
Durchflutungsgesetz\\
Quellenfreiheit $B$-Feld\\
Zusammenhang
\begin{align*}
    \varoiint_{A_\texttt{Hülle}} \vec{D} d \vec{A} & = Q_\texttt{eing.}                      \\
    \oint_\texttt{Rand} \vec{E} d \vec{s}          & = -\dfrac{d\Phi_\texttt{eing.}}{dt}     \\
    \oint_\texttt{Rand} \vec{H} d \vec{s}          & = I_\texttt{eing.} + I_\texttt{versch.} \\
    \varoiint_{A_\texttt{Hülle}} \vec{B} d \vec{A} & = 0
\end{align*}

\[
    \vec{D} = \varepsilon \cdot \vec{E} \qquad
    \vec{B} = \mu \cdot \vec{H}
\]

Bei isotropen Stoffen sind $\varepsilon$ u. $\mu$ Skalare:
\[
    \varepsilon = \varepsilon_0 \cdot \varepsilon_r \qquad \mu = \mu_0 \cdot \mu_r
\]

\subsection{Differentialform \rn{1}, \rn{2}}
\begin{align*}
    \operatorname{div} \vec{D} & = \rho                                           \\
    \operatorname{rot} \vec{E} & = -\dfrac{\partial \vec{B}}{\partial t}          \\
    \operatorname{rot} \vec{H} & = \vec{J} + \dfrac{\partial \vec{D}}{\partial t} \\
    \operatorname{div} \vec{B} & = 0
\end{align*}

\subsubsection*{Divergenz/Rotation/Gradient}

$\operatorname{div}$: macht aus einem Vektor ein Skalar.\\
$\operatorname{rot}$: bildet ein Vektor auf Vektorfeld ab.\\
$\operatorname{grad}$: bildet ein Skalar-/Gradientenfeld in ein Vektorfeld ab.
Zeigt Richtung stärkster Zunahme des Feldes.

\[
    \dfrac{\partial G_x}{\partial x} + \dfrac{\partial G_y}{\partial y} +
    \dfrac{\partial G_z}{\partial z} = \operatorname{div} \vec{G} = \nabla
    \cdot \vec{G}
\]

\begin{align*}
    \arraycolsep=1.0pt\def\arraystretch{2.0}
    \begin{pmatrix}
        \dfrac{\partial G_z}{\partial y} - \dfrac{\partial G_y}{\partial z} \\
        \dfrac{\partial G_x}{\partial z} - \dfrac{\partial G_z}{\partial x} \\
        \dfrac{\partial G_y}{\partial x} - \dfrac{\partial G_x}{\partial y} \\
    \end{pmatrix} & = \operatorname{rot} \vec{G} = \nabla \times \vec{G}                                 \\
    \arraycolsep=1.0pt\def\arraystretch{2.0}
    \begin{pmatrix}
        \dfrac{\partial G}{\partial x} \\
        \dfrac{\partial G}{\partial y} \\
        \dfrac{\partial G}{\partial z} \\
    \end{pmatrix} \qquad                                      & = \operatorname{grad} G = \nabla \cdot G
\end{align*}

\subsubsection*{Nabla Operator}

\[
    \nabla = \vec{\nabla} = \left( \dfrac{\partial G}{\partial x},
    \dfrac{\partial G}{\partial y}, \dfrac{\partial G}{\partial z} \right)
\]

Feldänderung bei Bewegung
\begin{align*}
    \Delta G & = \dfrac{\partial G}{\partial x} \Delta x + \dfrac{\partial G}{\partial y} \Delta y + \dfrac{\partial G}{\partial z} \Delta z \\
             & = dG = \operatorname{grad} G \cdot d \vec{s}
\end{align*}

\subsection{stationäre Felder}

\begin{align*}
    \nabla \cdot \vec{D}  & = \rho \qquad \vec{D} = \varepsilon \cdot \vec{E} \\
    \nabla \times \vec{E} & = 0                                               \\
    \nabla \times \vec{H} & = \vec{J} \qquad \vec{B} = \mu \cdot \vec{H}      \\
    \nabla \cdot \vec{B}  & = 0                                               \\
    \vec{J}               & = \kappa \vec{E}
\end{align*}

\subsection{statische Felder}
Elektrostatik: $\vec{J} = 0$\\
$\operatorname{rot} \vec{E} = 0$\\
Magnetostatik: $\vec{J} = const$
