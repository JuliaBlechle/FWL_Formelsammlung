\section{Antennen}

\subsection{Antennenkenngrößen}

% \begin{center}
    \resizebox{0.5\columnwidth}{!}{
        \begin{circuitikz}
            \draw(0,3.5) node[below]{$l$}
            to[R=$R_V$](5,3.5)
            to[R=$R_S$](5,1.5)
            to[L=$jX_A$](5,0)
            to[short](0,0) node[above]{$l'$}
            to[open, o-o](0,3.5);
            %\draw[-](-0.5,1.75) --(-0.5,1.25) ;
            \draw[latex-](1.75,1.75)--(0.5,1.75) node[left]{$\underline{Z}_A$};
        \end{circuitikz}
    }
% \end{center}


\subsubsection{Abgestrahlte Leistung}
\begin{align*}
    P_S & = \frac{1}{2}\cdot I_A^2 \cdot R_S
\end{align*}

\subsubsection{Verlustleistung}
\begin{align*}
    P_V & = \frac{1}{2}\cdot I_A^2\cdot R_V
\end{align*}

\subsubsection{Wirkungsgrad}
\begin{align*}
    \eta & = \frac{P_S}{P_S + P_V} = \frac{R_S}{R_S + R_V}
\end{align*}

\subsubsection{Richtfaktor und Gewinn}
\begin{align*}
     & D\left[\si{dB}\right]  &  & G\left[\si{dB}\right]  \\
     & d\left[\si{dBi}\right] &  & g\left[\si{dBi}\right] \\
     &                        &  & G  = \eta\cdot D       \\
     & d = 10\cdot\log_{10} D &  & g  = 10\cdot\log_{10}G
\end{align*}

\subsubsection{Wirksame Antennenfläche}
\begin{align*}
    A_W & = \frac{\lambda^2}{4\pi}\cdot G
\end{align*}

\subsection{Herz'scher Dipol}

\subsubsection{Nahfeld: $r \ll \lambda$}

Überwiegend \textbf{Blindleistungsfeld}, da $E$ zu $H$ $90^\circ$
phasenverschoben
\begin{empheq}[box=\fbox]{align*}
    H & = \vec{\Phi}\cdot\frac{I_0 \Delta l}{4\pi R^2}\cdot \sin\Theta                                                \\
    E & = \frac{I_0 \Delta l}{4\pi j \omega\varepsilon R^3}(2\vec{R} \cdot \cos\Theta + \vec{\Theta}\cdot \sin\Theta)
\end{empheq}

\begin{empheq}[box=\fbox]{align*}
    E_r       & = \frac{I l \lambda}{4\pi^2\varepsilon_0  c_0}\cdot \frac{1}{r^3} \cdot\cos\upsilon \cdot \sin(\omega t - \beta r) \\
    E_v       & = \frac{I l \lambda}{4\pi^2\varepsilon_0  c_0}\cdot \frac{1}{r^3} \cdot\sin\upsilon \cdot \sin(\omega t - \beta r) \\
    H_\varphi & = \frac{I l}{4\pi}\cdot \frac{1}{r^2}\cdot\sin\varphi\cdot\cos(\omega t -\beta r)
\end{empheq}

\subsubsection{Fernfeld: $r\gg\lambda$}

Überwiegend \textbf{Wirkleistungsfeld}, $\vec{S}$ nach außen somit Kugelwelle

\vspace{1ex}
mit $\eta = Z_{F0}$

\begin{empheq}[box=\fbox] {align*}
    H & = \vec{\Phi}\cdot j\frac{\beta I_0 \Delta l}{4\pi R}\cdot \sin\Theta                             \\
    E & = \vec{\Theta}\cdot j\frac{\beta I_0 \Delta l }{4\pi R} \cdot \sin\Theta \cdot\eta e^{-j\beta R}
\end{empheq}

\begin{empheq}[box=\fbox]{align*}
    E_r       & = 0                                                                                                           \\
    E_v       & = -\frac{I l }{2\varepsilon_0  c_0 \lambda}\cdot \frac{1}{r} \cdot\sin\upsilon \cdot \sin(\omega t - \beta r) \\
    H_\varphi & = -\frac{I l}{2\lambda}\cdot \frac{1}{r}\cdot\sin\upsilon\cdot\sin(\omega t -\beta r)
\end{empheq}

\subsubsection{Abgestrahlte Leistung im Fernfeld}
\begin{align*}
    P_{rad} = \frac{\eta {I_0}^2 \beta^2 (\Delta l')^2}{12\pi} = \frac{I_0^2\eta\pi}{3}\cdot \dfrac{\Delta l'^2}{\lambda^2}
\end{align*}

\subsection{Magnetischer Dipol}
\begin{align*}
    \Delta l & \rightarrow \beta\pi\ a^2
\end{align*}

\subsubsection{Fernfeld}
\begin{align*}
    E       & = \vec{\Phi}\cdot j\frac{\eta I_0\beta^2\pi a^2}{4\pi R}\cdot e^{-j\beta R}\cdot\sin\theta              \\
    H       & = \vec{\theta}\cdot j\frac{I_0\beta^2\pi a^2}{4\pi R}\cdot e^{-j\beta R}\cdot\sin\theta                 \\
    P_{rad} & = \frac{\eta I_0^2\beta^2(\beta\pi a^2)^2}{12\pi}=                                                      \\
            & = I_0^2\cdot\frac{\eta\pi}{3}\cdot(\frac{\beta\pi a^2}{\lambda})^2                                      \\
            & = I_0^2\cdot\frac{4}{3}\cdot Z_F\cdot\frac{\pi^5}{a^4\cdot\sqrt{\varepsilon_r^3}}\cdot(\frac{f}{c_0})^4
\end{align*}

\subsection{Bezugsantennen}
\[
    \boxed{g = 10 \cdot log(G)dB}
\]

mit $P_0$ : Eingangsleistung der Antenne

\begin{description}
    \item \textbf{\underline{G$\rightarrow$Bezugsantenne:}}

          Elementardipol  zu Kugelstrahler \[D = 1,50 \rightarrow g = 1,76\si{dBi}\]
          Halbwellendipol zu Kugelstrahler \[D = 1,64 \rightarrow g = 2,15\si{dBi}\]

    \item \textbf{\underline{EIRP}: Eqivalent \underline{Isoropic} Radiated Power}
          \[
              \text{EIRP} = P_0 \cdot G_i [dBi]
          \]

    \item \textbf{\underline{ERP}: Eqivalent Radiated Power (verlustloser Halbwellendipol)}
          \[
              \text{ERP} = P_0 \cdot G_d [dBd]
          \]
\end{description}

\subsection{Senden und Empfangen}
\begin{align*}
    P_E & = S_E\cdot A_W =                                                         \\
        & = P_S\cdot G_{TX}\cdot G_{RX}\cdot \left(\frac{\lambda}{4\pi r}\right)^2
\end{align*}
