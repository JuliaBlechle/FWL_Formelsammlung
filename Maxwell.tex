\section{Maxwell’schen Gleichungen}

\subsection{Intergralform \rn{1}, \rn{2}}
Gauß'sches Gesetz\\
Induktionsgesetz\\
Durchflutungsgesetz\\
Quellenfreiheit $B$-Feld\\
Zusammenhang
\begin{align*}
    \varoiint_{A_\texttt{Hülle}} \vec{D} d \vec{A} & = Q_\texttt{eing.}                                                                      \\
    \oint_\texttt{Rand} \vec{E} d \vec{s}          & = -\dfrac{d\Phi_\texttt{eing.}}{dt} = -\frac{d}{dt} \iint \vec{B} \, d\vec{A} = u_{ind} \\
    \oint_\texttt{Rand} \vec{H} d \vec{s}          & = I_\texttt{eing.} + I_\texttt{versch.}                                                 \\
    \varoiint_{A_\texttt{Hülle}} \vec{B} d \vec{A} & = 0
\end{align*}

\[
    \vec{D} = \varepsilon \cdot \vec{E} \qquad
    \vec{B} = \mu \cdot \vec{H}
\]

Bei isotropen Stoffen sind $\varepsilon$ u. $\mu$ Skalare:
\[
    \varepsilon = \varepsilon_0 \cdot \varepsilon_r \qquad \mu = \mu_0 \cdot \mu_r
\]

\subsection{Differentialform \rn{1}, \rn{2}}
\begin{align*}
    \opdiv \vec{D}             & = \rho                                           \\
    \operatorname{rot} \vec{E} & = -\dfrac{\partial \vec{B}}{\partial t}          \\
    \operatorname{rot} \vec{H} & = \vec{J} + \dfrac{\partial \vec{D}}{\partial t} \\
    \opdiv \vec{B}             & = 0
\end{align*}

\subsection{stationäre Felder (Gleichstrom)}

\begin{align*}
    \nabla \cdot \vec{D}  & = \rho \qquad \vec{D} = \varepsilon \cdot \vec{E} \\
    \nabla \times \vec{E} & = 0                                               \\
    \nabla \times \vec{H} & = \vec{J} \qquad \vec{B} = \mu \cdot \vec{H}      \\
    \nabla \cdot \vec{B}  & = 0                                               \\
    \vec{J}               & = \kappa \vec{E}
\end{align*}

% \subsection{statische Felder}
% Elektrostatik: $\vec{J} = 0$\\
% $\operatorname{rot} \vec{E} = 0$\\
% Magnetostatik: $\vec{J} = const$
