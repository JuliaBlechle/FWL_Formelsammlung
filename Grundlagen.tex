\section{Grundlagen}
\subsection{mathematische}
\subsubsection*{Divergenz/Rotation/Gradient}

$\opdiv$: Vektor $\rightarrow$ Skalar (Quelle / Senke)\\
$\oprot$: Vektor auf Vektorfeld ( rot = 0 für E. erhaltendes Feld)\\
$\opgrad$: Skalar-/Gradientenfeld $\rightarrow$ Vektorfeld (Richtung stärkster Feldänderung)
\begin{align*}
    \opdiv \vec{G} & = \nabla \cdot \vec{G}  & = & \dfrac{\partial G_x}{\partial x} + \dfrac{\partial G_y}{\partial y} + \dfrac{\partial G_z}{\partial z} \\
                   &                         &   & \begin{cases}
                                                       = 0 \quad\Rightarrow \textnormal{Volumen} \\
                                                       > 0 \quad\Rightarrow \textnormal{Quelle}  \\
                                                       < 0 \quad\Rightarrow \textnormal{Senke}
                                                   \end{cases}                                                              \\
    \oprot \vec{G} & = \nabla \times \vec{G} & = &
    \begin{pmatrix}
        \dfrac{\partial G_z}{\partial y} - \dfrac{\partial G_y}{\partial z} \\
        \dfrac{\partial G_x}{\partial z} - \dfrac{\partial G_z}{\partial x} \\
        \dfrac{\partial G_y}{\partial x} - \dfrac{\partial G_x}{\partial y}
    \end{pmatrix}                                                                                   \\
    \opgrad G      & = \nabla \cdot G        & = & \hspace{4.5ex}
    \begin{pmatrix}
        \dfrac{\partial G}{\partial x} \\
        \dfrac{\partial G}{\partial y} \\
        \dfrac{\partial G}{\partial z}
    \end{pmatrix}
\end{align*}


\subsection{Rechenregeln}
\begin{align*}
     & \nabla \cdot (\vec{f} \times \vec{g}) & = & \qquad (\nabla \times \vec{f})\cdot\vec{g} - (\nabla\times\vec{g})\cdot\vec{f} \\
     & \nabla \cdot (fg)                     & = & \qquad f(\nabla g) + g( \nabla f)                                              \\
     & \nabla \cdot (f\vec{g})               & = & \qquad g(\nabla\vec{f}) + f(\nabla\vec{g})                                     \\
     & \nabla \times (f\vec{g})              & = & \qquad \nabla f \times \vec{g} + f(\nabla\times\vec{g})                        \\
     & \oprot \opgrad f                      & = & \qquad 0 \Rightarrow\textnormal{Gradientenfeld Quellenfrei}                    \\
     & \opdiv \oprot \vec{f}                 & = & \qquad 0 \Rightarrow\textnormal{Wirbelfeld Quellenfrei}
\end{align*}


\subsubsection*{Nabla Operator}
\[
    \nabla = \vec{\nabla} = \left( \dfrac{\partial G}{\partial x},
    \dfrac{\partial G}{\partial y}, \dfrac{\partial G}{\partial z} \right)
\]

Feldänderung bei Bewegung
\begin{align*}
    \Delta G & = \dfrac{\partial G}{\partial x} \Delta x + \dfrac{\partial G}{\partial y} \Delta y + \dfrac{\partial G}{\partial z} \Delta z \\
             & = dG = \opgrad G \cdot d \vec{s}
\end{align*}

\subsection{Schnittwinkel zweier Vektoren}
\begin{align*}
    \vec{E} \cdot \vec{H} & = |\vec{E}| \cdot |\vec{H}| \cdot cos(\varphi)                         \\
    cos(\varphi)          & = \dfrac{E_x \cdot H_x + E_y \cdot H_y + E_z \cdot H_z}{|E| \cdot |H|}
\end{align*}

\subsection{Logarithmische Maße}

{\samepage
    \begin{itemize}
        \setlength\itemsep{0pt}
        \item $\si{dBm} \hat=  1\si{mW}$
        \item $\si{dB\mu} V \hat= 1\si{\mu V}$
        \item $\si{dBmV} \hat{=} 1mV$
        \item $\si{dBi} \rightarrow$ Isotropic
    \end{itemize}
}

\begin{description}
    \item Dezibel [dB]
          \begin{flalign*}
              X[dB]     & = 20 \cdot \log_{} \left( \dfrac{U_1}{U_2}\right) & X[dB] & = 10 \cdot \log_{} \left( \dfrac{P_1}{P_2}\right) & \\
              U_1       & = U_2 \cdot 10^{^X/20\si{dB}}                     & P_1   & = P_2 \cdot  10^{^X/10\si{dB}}                    & \\
              1 \si{dB} & \,\hat=                                           &       & 0,1151 \si{Np}                                    &
          \end{flalign*}

    \item Neper [Np]
          \begin{flalign*}
              X[Np]     & = \ln \left(\dfrac{U_1}{U_2}\right) & X[Np] & = \dfrac{1}{2} \cdot \ln \left(\dfrac{P_1}{P_2}\right) & \\
              U_1       & = U_2 \cdot e^{X}                   & P_1   & = P_2 \cdot  e^{2X}                                    & \\
              1 \si{Np} & \,\hat=                             &       & 8,686 \si{dB}                                          & % =\dfrac{20}{\ln 10} \cdot \ln \left( \dfrac{U_1}{U_2}\right)
          \end{flalign*}
\end{description}

\subsection{Randbedingung}
\begin{tabularx}{0.45\textwidth}{>{\hsize=.3\hsize}X|>{\hsize=.7\hsize}X}
    Dirichlet-RB & Funktion nimmt an den Rändern einen bestimmten Wert an (Bsp.: $\rho_r = 5V$) \\
    \hline
    Neumann-RB   & Die Normalableitung der Fkt. nimmt an den Rändern einen bestimmten Wert an   \\
\end{tabularx}

\subsection{Begriffe}
\begin{tabularx}{0.45\textwidth}{>{\hsize=.1\hsize}X|>{\hsize=.5\hsize}X|>{\hsize=.4\hsize}X}
           & Begriff           & Beschreibung \\
    \hline
    $\rho$ & Raumladungsdichte &              \\
\end{tabularx}




\subsection{Vergleich/Umrechnung}
\begin{tabularx}{0.45\textwidth}{>{\hsize=.46\hsize}X|>{\hsize=.27\hsize}X|>{\hsize=.27\hsize}X}
    Kart.                                                                                & Zyl.             & Kug.                            \\
    \specialrule{1.5pt}{0pt}{0pt}
    $x$                                                                                  & $r \cos \varphi$ & $r \sin \vartheta \cos \varphi$ \\
    \hline
    $y$                                                                                  & $r \sin \varphi$ & $r \sin \vartheta \sin \varphi$ \\
    \hline
    $z$                                                                                  & $z$              & $r \cos \vartheta$              \\
    \specialrule{1.5pt}{0pt}{0pt}
    $\sqrt{x^{2}+y^{2}}$                                                                 & $r$              &                                 \\
    \hline
    $\arctan \frac{y}{x}$                                                                & $\varphi$        &                                 \\
    \hline
    $z$                                                                                  & $z$              &                                 \\
    \hline
    $d x \cos \varphi+d y \sin \varphi$                                                  & $dr$             &                                 \\
    \hline
    $d y \cos \varphi-d x \sin \varphi$                                                  & $r d\varphi$     &                                 \\
    \hline
    $dz$                                                                                 & $dz$             &                                 \\
    \specialrule{1.5pt}{0pt}{0pt}
    $\sqrt{x^{2}+y^{2}+z^{2}}$                                                           &                  & $r$                             \\
    \hline
    $\arctan \frac{y}{x}$                                                                &                  & $\varphi$                       \\
    \hline
    $\arctan \frac{\sqrt{x^{2}+y^{2}}}{z}$                                               &                  & $\vartheta$                     \\
    \hline
    $d x \sin \vartheta \cos \varphi+d y \sin \vartheta \sin \varphi+d z \cos \vartheta$ &                  & $dr$                            \\
    \hline
    $d y \cos \varphi-d x \sin \varphi$                                                  &                  & $r \sin \vartheta d \varphi$    \\
    \hline
    $d x \cos \vartheta \cos \varphi+d y \cos \vartheta \sin \varphi-d z \sin \vartheta$ &                  & $r d \vartheta$                 \\
\end{tabularx}
