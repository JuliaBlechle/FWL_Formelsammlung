\section{Wellen}
\subsection{dÀlembertsche Gleichung (allg.)}

\begin{align*}
    \Delta \vec{E}-\kappa \mu \frac{\partial \vec{E}}{\partial t}-\varepsilon \mu \frac{\partial^{2} \vec{E}}{\partial t^{2}} & = \operatorname{grad} \frac{\rho}{\varepsilon} \\
    \Delta \vec{H}-\kappa \mu \frac{\partial \vec{H}}{\partial t}-\varepsilon \mu \frac{\partial^{2} \vec{H}}{\partial t^{2}} & = 0
\end{align*}

Isolator, ideales Dielektrikum, Nichtleiter $\kappa = 0$
\begin{align*}
    \Delta \vec{E} & =\varepsilon \mu \frac{\partial^{2} \vec{E}}{\partial t^{2}}+\operatorname{grad} \frac{\rho}{\varepsilon} \\
    \Delta \vec{H} & =\varepsilon \mu \frac{\partial^{2} \vec{H}}{\partial t^{2}}
\end{align*}

sehr gute Leiter
\begin{align*}
    \Delta \vec{E} & =\kappa \mu \frac{\partial \vec{E}}{\partial t}+\operatorname{grad} \frac{\rho}{\varepsilon} \\
    \Delta \vec{H} & =\kappa \mu \frac{\partial \vec{H}}{\partial t}
\end{align*}

\subsection{Helmholtz-Gleichungen (Frequenzbereich)}
\begin{align*}
    \Delta \underline{\vec{E}}-\left(\kappa \mu \cdot \mathrm{j} \omega-\varepsilon \mu \cdot \omega^{2}\right) \cdot \underline{\vec{E}} & = \operatorname{grad} \frac{\rho}{\varepsilon} \\
    \Delta \underline{\vec{H}}-\left(\kappa \mu \cdot \mathrm{j} \omega-\varepsilon \mu \cdot \omega^{2}\right) \cdot \underline{\vec{H}} & = 0
\end{align*}

\subsubsection{Zeitbereich}
\begin{align*}
    \Delta \vec{E}-\varepsilon \mu \frac{\partial^{2} \vec{E}}{\partial t^{2}} & =0 \\
    \Delta \vec{H}-\varepsilon \mu \frac{\partial^{2} \vec{H}}{\partial t^{2}} & =0
\end{align*}

\subsubsection{Frequenzbereich (harmonisch)}
\begin{align*}
    \Delta \underline{\vec{E}}+\varepsilon \mu \omega^{2} \cdot \underline{\vec{E}} & =0 \\
    \Delta \underline{\vec{H}}+\varepsilon \mu \omega^{2} \cdot \underline{\vec{H}} & =0
\end{align*}

%%%%%%%%%%%%%%%%%

\subsection{Wellenzahl}
Im Vakuum: $k_{0}=\frac{\omega}{c_{0}}$
\begin{align*}
    k & = \frac{\omega}{v_{p h}} = \frac{2 \pi f}{v_{p h}} = |\vec{k}|                                                              \\
      & = \frac{\omega \cdot n}{c_{0}} = n \cdot k_{0}=\frac{1}{\sqrt{\mu_{r} \cdot \varepsilon_{r}}} \cdot k_{0}=k_{r} \cdot k_{0}
\end{align*}

\subsection{Wellenlänge}
\begin{align*}
    \lambda & = \dfrac{2 \pi}{k} = \dfrac{\upsilon_{ph}}{f} = [m]                                                               \\
            & = \lambda_0 \cdot \dfrac{1}{\sqrt{\mu_r \cdot \varepsilon_r}} = \dfrac{\lambda_0}{n} = \dfrac{2 \pi}{n \cdot k_0} \\
            & \lambda_0 = \dfrac{c_0}{f} = \dfrac{2\pi}{k_0}
\end{align*}

\subsection{Phasengeschwindigkeit}
\[
    \dfrac{d z}{d t} = \upsilon_{ph} = \dfrac{\omega}{k} = \frac{1}{\sqrt{ \mu_r \mu_0 \varepsilon_r \varepsilon_0}} \qquad \upsilon_{ph,\textnormal{Medium} \leq c_0}
\]

\subsubsection{Gruppengeschwindigkeit}
\[
    \upsilon_{g} = \dfrac{d \omega}{d k} = \dfrac{\textnormal{Wegstück der Wellengruppe}}{\textnormal{Laufzeit der Wellengruppe}}
\]

\subsection{Feldwellenwiderstand}
allgemein, idealen Dielektrikum\\
freier Raum
\begin{align*}
    \underline{Z}_F    & = \dfrac{\underline{E}_{\textnormal{transversal}}}{\underline{H}_{\textnormal{transversal}}} = \sqrt{\dfrac{\mu}{\varepsilon}} \\
    \underline{Z}_{F0} & = \sqrt{\dfrac{\mu_0}{\varepsilon_0}} \approx 120\pi \Omega \approx 377 \Omega
\end{align*}

\subsection{Polarisation}
\begin{tabularx}{0.45\textwidth}{>{\hsize=.3\hsize}X|>{\hsize=.7\hsize}X}
    Lineare     & wenn der Endpunkt des E–Vektors eine Linie     \\
    \hline
    Elliptische & Endpunkt des E-Vektors eine Ellipse beschreibt
\end{tabularx}
