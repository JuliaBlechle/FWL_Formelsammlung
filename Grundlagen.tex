\section{Grundlagen}
\subsection{mathematische}
\subsubsection*{Divergenz/Rotation/Gradient}

$\opdiv$: macht aus einem Vektor ein Skalar.\\
$\operatorname{rot}$: bildet ein Vektor auf Vektorfeld ab.\\
$\opgrad$: bildet ein Skalar-/Gradientenfeld in ein Vektorfeld ab.
Zeigt Richtung stärkster Zunahme des Feldes.
\begin{align*}
    \opdiv \vec{G}             & = \nabla \cdot \vec{G} = \dfrac{\partial G_x}{\partial x} + \dfrac{\partial G_y}{\partial y} + \dfrac{\partial G_z}{\partial z} \\
                               & = 0 \quad\Rightarrow \textnormal{Volumen}                                                                                       \\
                               & > 0 \quad\Rightarrow \textnormal{Quelle}                                                                                        \\
                               & < 0 \quad\Rightarrow \textnormal{Senke}                                                                                         \\
    \operatorname{rot} \vec{G} & = \nabla \times \vec{G} =
    \begin{pmatrix}
        \dfrac{\partial G_z}{\partial y} - \dfrac{\partial G_y}{\partial z} \\
        \dfrac{\partial G_x}{\partial z} - \dfrac{\partial G_z}{\partial x} \\
        \dfrac{\partial G_y}{\partial x} - \dfrac{\partial G_x}{\partial y}
    \end{pmatrix}                                                                                          \\
    \opgrad G                  & = \nabla \cdot G = \hspace{6ex}
    \begin{pmatrix}
        \dfrac{\partial G}{\partial x} \\
        \dfrac{\partial G}{\partial y} \\
        \dfrac{\partial G}{\partial z}
    \end{pmatrix}
\end{align*}

\subsubsection*{Nabla Operator}
\[
    \nabla = \vec{\nabla} = \left( \dfrac{\partial G}{\partial x},
    \dfrac{\partial G}{\partial y}, \dfrac{\partial G}{\partial z} \right)
\]

Feldänderung bei Bewegung
\begin{align*}
    \Delta G & = \dfrac{\partial G}{\partial x} \Delta x + \dfrac{\partial G}{\partial y} \Delta y + \dfrac{\partial G}{\partial z} \Delta z \\
             & = dG = \opgrad G \cdot d \vec{s}
\end{align*}



\subsection{Randbedingung}
\begin{tabularx}{0.45\textwidth}{>{\hsize=.3\hsize}X|>{\hsize=.7\hsize}X}
    Dirichlet-RB & Funktion nimmt an den Rändern einen bestimmten Wert an (Bsp.: $\rho_r = 5V$) \\
    \hline
    Neumann-RB   & Die Normalableitung der Fkt. nimmt an den Rändern einen bestimmten Wert an   \\
\end{tabularx}

\subsection{Begriffe}
\begin{tabularx}{0.45\textwidth}{>{\hsize=.2\hsize}X|>{\hsize=.4\hsize}X|>{\hsize=.4\hsize}X}
           & Begriff           & Beschreibung \\
    \hline
    $\rho$ & Raumladungsdichte &              \\
\end{tabularx}


\subsection{Kartesische Koordinaten}
Einheitsvektoren:   $\quad \vec{e_{x}}, \vec{e_{y}}, \vec{e_{z}}$\\
Rechtssystem:       $\quad \vec{e_{x}} \times \vec{e_{y}}=\vec{e_{z}}$\\
Linienelemente:     $\quad d s=\sqrt{d x^{2}+d y^{2}+d z^{2}}$\\
Nabla Operator:     $\quad \nabla=\frac{\partial}{\partial x} \vec{e_{x}}+\frac{\partial}{\partial y} \vec{e_{y}}+\frac{\partial}{\partial z} \vec{e_{z}}$\\
Gradient:           $\quad \opgrad \varphi \equiv \nabla \varphi=\frac{\partial \varphi}{\partial x} \vec{e_{x}}+\frac{\partial \varphi}{\partial y} \vec{e_{y}}+\frac{\partial \varphi}{\partial z} \vec{e_{z}}$\\
Divergenz:          $\quad \opdiv \vec{D} \equiv \nabla \vec{D}=\frac{\partial D_{x}}{\partial x}+\frac{\partial D_{y}}{\partial y}+\frac{\partial D_{z}}{\partial z}$\\
Rotation:           $\quad \operatorname{rot} \vec{E} \equiv \nabla \times \vec{E} =$\\
$\left[\frac{\partial E_{z}}{\partial y}-\frac{\partial E_{y}}{\partial z}\right] \vec{e_{x}}+\left[\frac{\partial E_{x}}{\partial z}-\frac{\partial E_{z}}{\partial x}\right] \vec{e_{y}}+\left[\frac{\partial E_{y}}{\partial x}-\frac{\partial E_{x}}{\partial y}\right] \vec{e_{z}}$\\
Laplace Operator:   $\quad \Delta=\frac{\partial^{2} \ldots}{\partial x^{2}}+\frac{\partial^{2} \ldots}{\partial y^{2}}+\frac{\partial^{2} \ldots}{\partial z^{2}}$\\
\begin{align*}
     & \Delta \vec{E} = \opgrad \opdiv \vec{E}-\operatorname{rot} \operatorname{rot} \vec{E} =\Delta E_{x} \vec{e_{x}}+\Delta E_{y} \vec{e_{y}}+\Delta E_{z} \vec{e_{z}}=                                                                                                                                              \\
     & = \left[\frac{\partial^{2} E_{x}}{\partial x^{2}}+\frac{\partial^{2} E_{x}}{\partial y^{2}}+\frac{\partial^{2} E_{x}}{\partial z^{2}}\right] \vec{e_{x}}+\left[\frac{\partial^{2} E_{y}}{\partial x^{2}}+\frac{\partial^{2} E_{y}}{\partial y^{2}}+\frac{\partial^{2} E_{y}}{\partial z^{2}}\right] \vec{e_{y}} \\
     & + \left[\frac{\partial^{2} E_{z}}{\partial x^{2}}+\frac{\partial^{2} E_{z}}{\partial y^{2}}+\frac{\partial^{2} E_{z}}{\partial z^{2}}\right] \vec{e_{z}}
\end{align*}


\subsection{Zylinderkoordinaten}
Variablen:          $\quad r, \alpha, z$\\
Einheitsvektoren:   $\quad \vec{e_{r}}, \vec{e_{\alpha}}, \vec{e_{z}} \quad$\\
Rechtssystem: $\vec{e_{r}} \times \vec{e_{\alpha}}=\vec{e_{z}}$\\
% Zusammenhang mit rechtwinkligen Koordinaten:
% \begin{align*}
%     x&=r \cos \alpha \quad r=\sqrt{x^{2}+y^{2}} \quad d r=d x \cos \alpha+d y \sin \alpha\\
%     y&=r \sin \alpha \quad \alpha=\arctan \frac{y}{x} \quad r d \alpha=d y \cos \alpha-d x \sin \alpha\\
%     z&=z \quad z=z \quad d z=d z
% \end{align*}
Linienelemente:     $\quad d s=\sqrt{d r^{2}+r^{2} d \alpha^{2}+d z^{2}}$\\
Volumenelemente:    $\quad d v=r d r d \alpha d z$\\
Nabla Operator:     $\quad \nabla=\frac{\partial}{\partial r} \vec{e_{r}}+\frac{1}{r} \frac{\partial}{\partial \alpha} \vec{e_{\alpha}}+\frac{\partial}{\partial z} \vec{e_{z}}$\\
Gradient:           $\quad \opgrad \varphi \equiv \nabla \varphi=\frac{\partial \varphi}{\partial r} \vec{e_{r}}+\frac{1}{r} \frac{\partial \varphi}{\partial \alpha} \vec{e_{\alpha}}+\frac{\partial \varphi}{\partial z} \vec{e_{z}}$\\
Divergenz:          $\quad \opdiv \vec{D} \equiv \nabla \vec{D}=\frac{1}{r} \frac{\partial\left(r \vec{D}_{r}\right)}{\partial r}+\frac{1}{r} \frac{\partial \vec{D}_{\alpha}}{\partial \alpha}+\frac{\partial \vec{D}_{z}}{\partial z}$\\
Rotation:           $\quad \operatorname{rot} \vec{E} \equiv \nabla \times \vec{E}=\left[\frac{1}{r} \frac{\partial E_{z}}{\partial \alpha}-\frac{\partial E_{\alpha}}{\partial z}\right] \vec{e_{r}}+\left[\frac{\partial E_{r}}{\partial z}-\frac{\partial E_{z}}{\partial r}\right] \vec{e_{\alpha}}+\left[\frac{1}{r} \frac{\partial\left(r E_{\alpha}\right)}{\partial r}-\frac{1}{r} \frac{\partial E_{r}}{\partial \alpha}\right] \vec{e_{z}}$\\
Laplace Operator:   $\quad \Delta = \frac{1}{r}\frac{\partial \left(r \frac{\partial \ldots}{\partial r}\right)}{\partial r} + \frac{1}{r^2}\frac{\partial^2 \ldots}{\partial \alpha^2} + \frac{\partial^2 \ldots}{\partial z^2}$\\
\begin{align*}
    \vec{E} = & \left[\Delta E_{r}-\frac{2}{r^{2}} \frac{\partial E_{\alpha}}{\partial \alpha}-\frac{E_{r}}{r^{2}}\right] \vec{e_{r}}                                                  \\
              & +\left[\Delta E_{\alpha}+\frac{2}{r^{2}} \frac{\partial E_{r}}{\partial \alpha}-\frac{E_{\alpha}}{r^{2}}\right] \vec{e_{\alpha}}+\left[\Delta E_{z}\right] \vec{e_{z}}
\end{align*}


\subsection{Kugelkoordinaten}
Variablen:          $\quad r, \vartheta, \alpha$\\
Einheitsvektoren:   $\quad \vec{e_{r}}, \vec{e_{\vartheta}}, \vec{e_{\alpha}}$\\
Rechtssystem:       $\quad \vec{e_{r}} \times \vec{e_{\vartheta}}=\vec{e_{\alpha}}$\\
% Zusammenhang mit rechtwinkligen Koordinaten:
% \begin{align*}
%     x&=r \sin \vartheta \cos \alpha \quad r=\sqrt{x^{2}+y^{2}+z^{2}} \quad d r=d x \sin \vartheta \cos \alpha+d y \sin \vartheta \sin \alpha+d z \cos \vartheta\\
%     y&=r \sin \vartheta \sin \alpha \quad \alpha=\arctan \frac{y}{x} \quad r \sin \vartheta d \alpha=d y \cos \alpha-d x \sin \alpha\\
%     z&=r \cos \vartheta \quad \vartheta=\arctan \frac{\sqrt{x^{2}+y^{2}}}{z} \quad r d \vartheta=d x \cos \vartheta \cos \alpha+d y \cos \vartheta \sin \alpha-d z \sin \vartheta
% \end{align*}
Linienelement:      $\quad d s=\sqrt{d r^{2}+r^{2} \sin ^{2} \vartheta d \alpha^{2}+r^{2} d \vartheta^{2}}$\\
Volumenelement:     $\quad d v=r^{2} \sin \vartheta d r d \vartheta d \alpha$\\
Nabla Operator:     $\quad \nabla=\frac{\partial}{\partial r} \vec{e_{r}}+\frac{1}{r} \frac{\partial}{\partial \vartheta} \vec{e_{\vartheta}}+\frac{1}{r \sin \vartheta} \frac{\partial}{\partial \alpha} \vec{e_{\alpha}}$\\
Gradient:           $\quad \opgrad \varphi \equiv \nabla \varphi=\frac{\partial \varphi}{\partial r} \vec{e_{r}}+\frac{1}{r} \frac{\partial \varphi}{\partial \vartheta} \vec{e_{\vartheta}}+\frac{1}{r \sin \vartheta} \frac{\partial \varphi}{\partial \alpha} \vec{e_{\alpha}}$\\
Divergenz:          $\quad \opdiv \vec{D} \equiv \nabla \vec{D}=\frac{1}{r^{2}} \frac{\partial\left(r^{2} D_{r}\right)}{\partial r}+\frac{1}{r \sin \vartheta} \frac{\partial\left(\sin \vartheta \cdot D_{\vartheta}\right)}{\partial \vartheta}+\frac{1}{r \sin \vartheta} \frac{\partial D_{\alpha}}{\partial \alpha}$\\
Rotation:
\begin{align*}
    \operatorname{rot} \vec{E} & \equiv \nabla \times \vec{E}= \frac{1}{r \sin \vartheta}\left[\frac{\partial\left(\sin \vartheta \cdot E_{\alpha}\right)}{\partial \vartheta}-\frac{\partial E_{\vartheta}}{\partial \alpha}\right] \vec{e_{r}}                                                                                 \\
    +                          & \frac{1}{r}\left[\frac{1}{\sin \vartheta} \frac{\partial E_{r}}{\partial \alpha}-\frac{\partial r E_{\alpha}}{\partial r}\right] \vec{e_{\vartheta}}+\frac{1}{r}\left[\frac{\partial\left(r E_{\vartheta}\right)}{\partial r}-\frac{\partial E_{r}}{\partial \vartheta}\right] \vec{e_{\alpha}}
\end{align*}
Laplace Operator:   $\quad \Delta=\frac{1}{r^{2}} \frac{\partial\left(r^{2} \frac{\partial . .}{\partial r}\right)}{\partial r}+\frac{1}{r^{2} \sin \vartheta} \frac{\partial\left(\sin \vartheta \frac{\partial . \ddot{ }}{\partial \vartheta}\right)}{\partial \vartheta}+\frac{1}{r^{2} \sin ^{2} \vartheta} \frac{\partial^{2} . .}{\partial \alpha^{2}}$\\
Laplace Operator in Kugelkoordinaten, angewandt auf einen Vektor:
\begin{align*}
    \Delta \vec{E} & =\left[\Delta E_{r}-\frac{2}{r^{2}} E_{r}-\frac{2}{r^{2} \sin \vartheta} \frac{\partial\left(\sin \vartheta \cdot E_{\vartheta}\right)}{\partial \vartheta}-\frac{2}{r^{2} \sin \vartheta} \frac{\partial E_{\alpha}}{\partial \alpha}\right] \vec{e_{r}}        \\
                   & +\left[\Delta E_{\vartheta}-\frac{E_{\vartheta}}{r^{2} \sin ^{2} \vartheta}+\frac{2}{r^{2}} \frac{\partial E_{r}}{\partial \vartheta}-\frac{2 \cot \vartheta}{r^{2} \sin \vartheta} \frac{\partial E_{\alpha}}{\partial \alpha}\right] \vec{e_{\vartheta}}       \\
                   & +\left[\Delta E_{\alpha}-\frac{E_{\alpha}}{r^{2} \sin ^{2} \vartheta}+\frac{2}{r^{2} \sin \vartheta} \frac{\partial E_{r}}{\partial \alpha}+\frac{2 \cot \vartheta}{r^{2} \sin \vartheta} \frac{\partial E_{\vartheta}}{\partial \alpha}\right] \vec{e_{\alpha}}
\end{align*}



\subsection{Vergleich/Umrechnung}
\begin{tabularx}{0.45\textwidth}{>{\hsize=.46\hsize}X|>{\hsize=.27\hsize}X|>{\hsize=.27\hsize}X}
    Kart.                                                                              & Zyl.            & Kug.                           \\
    \specialrule{1.5pt}{0pt}{0pt}
    $x$                                                                                & $r \cos \alpha$ & $r \sin \vartheta \cos \alpha$ \\
    \hline
    $y$                                                                                & $r \sin \alpha$ & $r \sin \vartheta \sin \alpha$ \\
    \hline
    $z$                                                                                & $z$             & $r \cos \vartheta$             \\
    \specialrule{1.5pt}{0pt}{0pt}
    $\sqrt{x^{2}+y^{2}}$                                                               & $r$             &                                \\
    \hline
    $\arctan \frac{y}{x}$                                                              & $\alpha$        &                                \\
    \hline
    $z$                                                                                & $z$             &                                \\
    \hline
    $d x \cos \alpha+d y \sin \alpha$                                                  & $dr$            &                                \\
    \hline
    $d y \cos \alpha-d x \sin \alpha$                                                  & $r d\alpha$     &                                \\
    \hline
    $dz$                                                                               & $dz$            &                                \\
    \specialrule{1.5pt}{0pt}{0pt}
    $\sqrt{x^{2}+y^{2}+z^{2}}$                                                         &                 & $r$                            \\
    \hline
    $\arctan \frac{y}{x}$                                                              &                 & $\alpha$                       \\
    \hline
    $\arctan \frac{\sqrt{x^{2}+y^{2}}}{z}$                                             &                 & $\vartheta$                    \\
    \hline
    $d x \sin \vartheta \cos \alpha+d y \sin \vartheta \sin \alpha+d z \cos \vartheta$ &                 & $dr$                           \\
    \hline
    $d y \cos \alpha-d x \sin \alpha$                                                  &                 & $r \sin \vartheta d \alpha$    \\
    \hline
    $d x \cos \vartheta \cos \alpha+d y \cos \vartheta \sin \alpha-d z \sin \vartheta$ &                 & $r d \vartheta$                \\
\end{tabularx}
