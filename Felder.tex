\section{Felder}

\subsection{Elektrostatik}
ist ein wirbelfreies Feld. Elek. Ladungen sind Quellen des Feldes.
\begin{align*}
    \opdiv \vec{D} = \nabla \cdot \vec{D}  & = \rho       \qquad          \vec{D} = \varepsilon \vec{E} \\
    \oprot \vec{E} = \nabla \times \vec{E} & = 0 \quad = \oprot \opgrad E                               \\
    \vec{E}                                & =- \opgrad \varphi
\end{align*}
\subsubsection{Potetialgleichung}
\[
    \opdiv \opgrad \varphi = - \dfrac{\rho}{\varepsilon}
\]
\paragraph{$\qquad \Rightarrow$ Poisson-Gleichung}
mit $\rho = 0 \Rightarrow$ \textbf{Laplace-Gleichung}
\begin{align*}
    \Delta \varphi + \underbrace{ \dfrac{\opgrad \varepsilon \cdot \opgrad \varphi}{\varepsilon}}_{= 0\texttt{, wenn homogen}}
     & = - \dfrac{\rho (x, y, z)}{\varepsilon} \\
    \frac{d^2 \varphi}{d x^2} + \frac{d^2 \varphi}{d y^2} + \frac{d^2 \varphi}{d z^2}
     & = - \dfrac{\rho (x, y, z)}{\varepsilon}
\end{align*}

\subsection{Skineffekt}

\begin{description}
    \item Leitfähigkeit:
          \[
              \delta = \frac{1}{\sqrt{\pi\mu\sigma f}} = \sqrt{\frac{2}{\omega\mu\sigma}} \\
          \]

    \item Widerstand:
          \[
              R = \frac{1}{\sigma} - \frac{l}{A_{\texttt{eff}}}
          \]

    \item Effiktive Fläche:
          \begin{align*}
              A_{\texttt{eff}} & = A_{\texttt{ges}} - A_{\sigma} = R^2\pi-(R-\sigma)^2\pi \\
                               & =2\cdot \pi \delta \left( R-\dfrac{\delta }{2}\right)
          \end{align*}
\end{description}

Wenn die Länge nicht gegeben ist oder nach Wieviel \% nimmt der Widerstand bei
einer bestimmten Frequenz, kann dies mit der folgenden Formel berechnet werden:

\begin{description}
    \item Bessel-Funktion:
          \begin{align*}
              \frac{R_{AC}}{R_{DC}} & =
              \begin{dcases}
                  1 + \frac{1}{3}x^3              & \text{für} \qquad x > 1 \\
                  x + \frac{1}{4} + \frac{3}{64x} & \text{für} \qquad x < 1 \\
              \end{dcases} \\
              \frac{R_{AC}}{R_{DC}} & =
              \begin{dcases}
                  x^2\left( 1-\frac{x^4}{6} \right)                               & \text{für} \qquad x > 1 \\
                  x- \frac{1}{4} + \frac{3}{64x} + \frac{1}{4} + \frac{3}{128x^2} & \text{für} \qquad x < 1 \\
              \end{dcases}
          \end{align*}
          \[
              \boxed{x=\frac{R}{2\delta}}
          \]
\end{description}
